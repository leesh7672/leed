\chapter*{文法}
夫單詞也者、空詞與充詞之大名。空詞也者、單詞之未備字與音而具義。充詞也者、單詞之均有字與音與義。
漢文單詞品以體詞・謂詞・狀詞・束詞・縛詞・助詞・歎詞。
體詞咸語生物・死物・地物・時物。古分體詞以名詞・數詞・量詞、其能不分、名之以體詞也。
體詞可成主部・謂部・賓部・補部・狀部。
謂詞皆語體詞所語之態。古分謂詞以動詞與形容詞、其能不分、名以謂詞也。
謂詞維謂部寔成。
狀詞都爲定部・狀部。一統單詞・單節・單句・複句者曰束詞。語體詞單節之格曰縛詞。助詞也者、語氣之謂也。
助詞維助部實爲。
夫單句應有謂部焉。
單句先有謂部、後可從謂部而具主部・狀部・賓部・補部。
單句先備主部・賓部・補部、後得以有其定部。
若單句先有狀部、後有其狀部。

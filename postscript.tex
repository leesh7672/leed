\chapter{句法}
\par 夫單詞者、空詞與充詞之謂也。單詞之不具文字與語音曰空詞、否則謂之充詞。
漢文單詞品以體詞・指詞・定詞・算詞・數詞・量詞・束詞・格詞・謂詞・助詞・時詞・成詞。
\par 
數詞之與算節結也、則爲數節。算詞之與數節結也、則爲算節。
量詞之與數節結也、則爲量節。體詞之與量節結也、則爲體節。
定詞之與體節結也、則爲定節。指詞之與定節結也、則爲指節。
束詞之與指節結也、則爲束節。束節之與指節結也、則爲指節。
格詞之與指節結也、則爲格節。謂詞之與格節結也、則爲謂節。
助詞之與謂節結也、則爲助節。助節之與指節結也、則爲助節。
助詞之與助節結也、則爲助節。時詞之與助節結也、則爲時節。
成詞之與時節結也、則爲成節。
\chapter{愼引馨書}
\par 中國社會科學院. (2017). \textbl{現代漢語詞典}. 商務印書館.
\par \underline{梅廣}. (2017). \textbl{고대중국어 문법론}. 한국문화사. (本際2017年而刊)
\par \underline{鄧思穎}. (2021). \textbl{중국어 형식 통사론}. 고려대학교 출판문화원. (本際2019年而刊)
\par \underline{金琮鎬}. (2019). \textbl{한문 해석 공식}. 한티미디어.
\par Hornby, A. S. (2015). \textit{Oxford Advanced Learner's Dictionary}. Oxford University Press.

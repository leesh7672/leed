\chapter*{跋言}
此書所以拒舊法、從新語學所說而作辭書者、徒以憫苦然而習漢文、而虛光陰者也。
然公校便不之敎、顧令生握浮雲、恐愚直而從之者不得以識余所言也。
如\underline{鄧思頴}(2019/2021, pp. 56-77)所說、
今代漢語單詞分以體詞\textsubscript{N}・定詞\textsubscript{A}・量詞\textsubscript{Cl}
・數詞\textsubscript{Num}・指詞\textsubscript{D}・謂詞\textsubscript{V}・
狀詞\textsubscript{Adv}・助詞\textsubscript{v}・補詞\textsubscript{C}
・氣詞\textsubscript{F}・介詞\textsubscript{P}・束詞\textsubscript{Co}。
古代漢語單詞因今代漢語而分也。
\chapter*{馨書}
此書序文・跋文・內文愼引馨書如此。
\par 中國社會科學院. (2017). \textbl{現代漢語詞典}. 商務印書館.
\par \underline{梅廣}. (2017). \textbl{고대중국어 문법론}. 한국문화사. (本際2017年而刊)
\par \underline{鄧思穎}. (2021). \textbl{중국어 형식 통사론}. 고려대학교 출판문화원. (本際2019年而刊)
\par \underline{金琮鎬}. (2019). \textbl{한문 해석 공식}. 한티미디어.
\par Hornby, A. S. (2015). \textit{Oxford Advanced Learner's Dictionary}. Oxford University Press.

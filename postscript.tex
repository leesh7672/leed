\chapter{句法}
凡詞品以體詞・量詞・數詞・謂詞・輕謂詞・指詞・定詞・狀詞・介詞・連詞・助詞・補詞。
凡句式之始直一座之有也。
句式之一得新補詞座也、方可復得新補詞於其補詞座前。
句式之一得新補詞座也、便得助詞於其補詞座後。
句式之一得新助詞座也、便得輕謂詞於其助詞座前。
句式之一得新輕謂詞座也、便得指詞於其輕謂詞座前、可復得新輕謂詞於其輕謂詞座後。
句式之一得新輕謂詞座、而不得新輕謂詞座也、便得謂詞於其經謂詞座前。
句式之一得新謂詞座也、方可得介詞於其謂詞座後。
句式之一得新謂詞座也、而不得新介詞座也、便可得補詞於其謂詞座後。
句式之一得新介詞座也、便可得指詞座於其介詞座前後也。
句式之一得指詞座、而不在介詞座前也、便可得體詞座於指詞座後。
句式之一得指詞座、而在介詞座前也、便罕得體詞座於指詞座後。
句式之一得體詞座、便可得量詞座於體詞座後、可得定詞座於其前、而可得補詞座於前後。
句式之一得量詞座、便可得數詞座於量詞座前。
如此而成句式、則足以該品之詞盛諸座也。

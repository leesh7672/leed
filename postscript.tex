\chapter*{跋言}
此書所以拒舊法、從新語學所說而作辭書者、徒以憫苦然而習漢文、而虛光陰者也。
然公校便不之敎、顧令生握浮雲、恐愚直而從之者不得以識余所言也。
如鄧思頴(2019/2021, pp. 56-77)所說、漢語單詞分以體詞\textsubscript{N}・定詞\textsubscript{A}・量詞\textsubscript{Cl}・數詞\textsubscript{Num}・指詞\textsubscript{D}・謂詞\textsubscript{V}・狀詞\textsubscript{Adv}・助詞\textsubscript{v}・補詞\textsubscript{C}・氣詞\textsubscript{F}・冠詞\textsubscript{P}・束詞\textsubscript{Co}。
\chapter{馨典}
\paragraph\par 中國社會科學院. (2017). \textit{現代漢語詞典}. 商務印書館.
\paragraph\par 梅廣. (2017). \textit{고대중국어 문법론}. 한국문화사. (本刊以2017。)
\paragraph\par 鄧思穎. (2021). \textit{중국어 형식 통사론}. 고려대학교 출판문화원. (本刊以2019。)
\paragraph\par 金琮鎬. (2019). \textit{한문 해석 공식}. 한티미디어.
\paragraph\par Hornby, A. S. (2015). \textit{Oxford Advanced Learner's Dictionary}. Oxford University Press.

\chapter*{跋言}
夫單詞者,空詞與充詞之謂也。單詞之不具文字與語音曰空詞,否則謂之充詞。
漢文單詞品以體詞、指詞、定詞、數詞、量詞、謂詞、狀詞、事態助詞、時態助詞、格物助詞、語氣助詞、補體助詞、補句助詞。
單詞之語生物、死物、時物、地物也,曰體詞。物詞道而定局曰定詞。指四物曰指詞。
數詞與量詞如所周知。謂詞者,動詞與靜謂詞之謂也。
明言謂詞所說之時,曰狀詞。事態詞又曰輕動詞,舊曰輕量動詞。
事態助詞、時態助詞、格物助詞、語氣助詞、補體助詞、補句助詞甚虛,難說其義。

\chapter*{馨書}
此書序文・跋文・內文愼引馨書如此。
\par 中國社會科學院. (2017). \textbl{現代漢語詞典}. 商務印書館.
\par \underline{梅廣}. (2017). \textbl{고대중국어 문법론}. 한국문화사. (本際2017年而刊)
\par \underline{鄧思穎}. (2021). \textbl{중국어 형식 통사론}. 고려대학교 출판문화원. (本際2019年而刊)
\par \underline{金琮鎬}. (2019). \textbl{한문 해석 공식}. 한티미디어.
\par Hornby, A. S. (2015). \textit{Oxford Advanced Learner's Dictionary}. Oxford University Press.

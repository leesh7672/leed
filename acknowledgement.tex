余所能以編此書者,以有應結草所報之恩也。
余始編此書之時,猶未逾大學本科生。
故余所未曾學者甚蕃,看書而頻借其所語,不語其所由焉。
余生於槿邦,幼時雖習漢字於初校,未懷志才也。
然中學有恩師姜氏者執日用漢語,教太白詩而動吾心,有恩師朴氏者執文言,而矯余病文。
因故退於高校,妄思此身之分,圖進首爾大學,而習高校漢文於教育放送金夫子。
若未蒙其教,詩文類童,必為左右笑。
雖竭力而習漢文,而遠東文、英文、算學之類,斯謀重舉,竟不為余所行。
時橫海遊覽扶桑國,欲讀和文於京都大學,而國事之亂奪此志矣。
以是讀於韓國放送通信大學,緣日本學科之不教古典文學,乃修中國語言文學。
余所以修中國語學者,徒以天之鞏此志也。
是以謝之如此。
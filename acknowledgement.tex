余所能以編此書者以有應結草所報之恩也。余始編此書之時,猶未逾大學本科生。故,余所未曾學者頗多,看書而頻借其所語,不語其所由焉。

余生於槿邦,幼時雖習漢字於初校,猶未懷志才,而中學有恩師姜氏者執日用漢語教太白詩而動吾心,有恩師朴氏者執文言而矯余病文。

因故退於高校,妄思此身之分,圖進首爾大學,而習高校漢文於教育放送金夫子。若未蒙其教,詩文類童必為左右笑。
雖竭力而習漢文而遠東文、英文、算學之類斯謀重舉,而竟不為余所行。
時,橫海覽扶桑欲讀和文於京都大學,而國事之亂奪此志矣。

余所以修中國語學者徒以天之鞏此志也。
以是讀於韓國放送通信大學。緣日本學部門之不教古典文學,乃修中國語言文學。
於是謝之如此。
余所能以編此書者,以有應結草所報之恩也。
余始編此書之時,猶未逾本科生。
故余所未學甚蕃,看書而頻借其所語,不語其所由焉。
生於槿邦,幼時雖習漢字於初校,未懷志才也。
然中學有恩師姜氏者執日用漢語,敎太白詩而動吾心,有恩師朴氏者執漢文,而矯余病文。
因故退於高校,不知此身之分,圖進首都大學,而習高校漢文於敎養放送金夫子。
若未蒙其敎,詩文類童,必爲左右笑。
雖盡心習漢文,而遠東文、英文、算學之類,斯謀重舉,竟不爲余所行。
時横海遊覽扶桑,欲讀和文於京都大學,而國事之亂奪其志矣。
以是讀於韓國放送通信大學,因日本學科不教古文學,乃修中語中文學。
小生修中語學,徒以天道竊所行也。
是以謝之如此。

余所能以編此書者,以有應結草所報之恩也。余始編此書之時,猶未逾大學本科生,故余所未曾學者頗多,看書而頻借其所語,不語其所由焉。
余生於槿邦,幼時雖習漢字於初校,猶未懷志才,中學有恩師姜氏者執日用漢語,教太白詩而動吾心,有恩師朴氏者執文言,而矯余病文。
因故退於高校,妄思此身之分,圖進首爾大學,而習高校漢文於教育放送金夫子。若未蒙其教,詩文類童,必為左右笑。
雖竭力而習漢文,而遠東文、英文、算學之類,斯謀重舉,竟不為余所行。時橫海遊覽扶桑國,欲讀和文於京都大學,而國事之亂奪此志矣。
以是讀於韓國放送通信大學,緣日本學部門之不教古典文學,乃修中國語言文學。
余所以修中國語學者,徒以天之鞏此志也,於是謝之如此。
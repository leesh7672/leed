\documentclass[a5paper,10pt]{report}
\usepackage{luatexko}
\usepackage{geometry}
\usepackage{multicol}
\usepackage{setspace}
\usepackage{wrapfig}
\usepackage{xcolor}
\usepackage{graphicx}
\usepackage{etoolbox}
\usepackage{titlesec}
\usepackage{zhnumber}
\usepackage{cancel}
\usepackage[linguistics]{forest}
\usepackage{tikz}

\titleformat{\chapter}{\fontsize{18pt}{18pt}\selectfont}{}{0pt}{}
\titleformat{\section}{\fontsize{14pt}{14pt}\selectfont}{}{0pt}{}

\definecolor{gold}{rgb}{0.4,0.3,0.2}
\definecolor{deepgold}{rgb}{0.3,0.2,0.1}
\graphicspath{{./images/}}

\hanjabyhanjafont=1
\setmainfont[InterCharStretch=2pt,Vertical=Alternates,RawFeature=vertical]{HCR Batang Lvt}
\setmainhangulfont[InterCharStretch=2pt,Vertical=Alternates,RawFeature=vertical]{HCR Batang Lvt}
\setmainhanjafont[InterCharStretch=2pt,Vertical=Alternates,RawFeature=vertical]{HCR Batang Lvt}

\newcommand{\zero}{∅}
\newcommand{\explain}[2]{{$\bullet$}〔{#1}〕{#2}}
\newcommand{\entry}[4]{\par\textcolor{gold}{#1}\textsuperscript{#2} {#3}}
\newcommand{\cref}[1]{\textcolor{gold}{#1}}

\geometry{top=10mm,left=10mm,bottom=25mm,right=10mm}
\verticaltypesetting
\begin{document}
\begin{titlepage}
\hfill
\vfill
\hspace{-40mm}
\includegraphics[angle=90,width=250mm]{cover.png}
\vfill
{\fontsize{14pt}{14pt}\selectfont\textcolor{gold}{李世鎬}\hspace{24pt}編}
\vspace{24pt}\newline
{\fontsize{36pt}{36pt}\selectfont\textcolor{gold}{\hspace{48pt}古典漢文敎學辭書}}
\vfill
\end{titlepage}
\newpage
\chapter*{首}
\onehalfspacing
\chapter*{序文}
華夏・雞林・扶桑・南越之中、書生齊嘗與漢文流通。
夫欲習漢文也、古以字典爲寶及今。
字典雖有字彙之釋、而未有詞。
各國有其辭書、以諺釋文、
不識其國之諺、畫中之餅也爾。
內文各項或取諸漢籍、或從古道而新製。
各項順之康熙字典。
同源之詞同項而登、異源之詞別項而載。
以文釋文、詳述其用、弱通漢文、皆可以讀。
此書之爲江湖諸賢所修與天下諸儒所寶甚遙遠。
\chapter*{文法}
詞品以成詞・助詞・狀詞・態詞・謂詞・縛詞・結詞・指詞
・質詞・體詞・度詞・算詞・數詞之大名也。
\cref{父}・\cref{堯}・\cref{犬}・\cref{葡萄}・\cref{石}之類曰體詞。
\cref{二}・\cref{旬}・\cref{百}・\cref{萬}之類曰數詞。
\cref{又}・\cref{有}之類、曰算詞。
\cref{日}・\cref{月}・\cref{匹}之類曰度詞。
\cref{此}・\cref{何}・\cref{誰}之類曰指詞。
\cref{與}・\cref{及}・\cref{兼}之類曰結詞。
\cref{於}・\cref{猶}・\cref{曰}之類曰縛詞。
\cref{爲}・\cref{去}・\cref{有}・\cref{美}・\cref{因}之類曰謂詞。
\cref{相}・\cref{使}・\cref{遭}之類曰態詞。
\cref{也}・\cref{矣}・\cref{乎}・\cref{而}之類曰助詞。
\cref{大凡}・\cref{若}・\cref{雖}・\cref{然}・\cref{而}之類曰成詞。
\cref{不}・\cref{皆}・\cref{徒}・\cref{所}・\cref{以}之類曰狀詞。

大凡詞組時析以核部與補部、時復析以冠部與心部、心部之成似詞組。
詞組與心部之內先有冠部、然後心部。
助詞組與度詞組之外、先有核部、後有補部。
助詞組與度詞組之內、先有補部、後有核部。
數詞可以爲數詞組核部、算詞可以爲算詞組核部、度詞可以爲度詞組核部
束詞可以爲束詞組核部、體詞可以爲體詞組核部、指詞可以爲指詞組核部、
結詞可以爲結詞組核部、縛詞可以爲縛詞組核部、謂詞可以爲謂詞組核部、
態詞可以爲態詞組核部、助詞可以爲助詞組核部、成詞可以爲成詞組核部。
質詞可以爲體詞組冠部、狀詞可以爲態詞組冠部。

數詞組可以爲算詞組補部、可以爲度詞組補部。
算詞組可以爲數詞組補部。
度詞組可以爲體詞組補部。
體詞組可以爲指詞組補部、可以爲結詞組冠部。
結詞組可以爲指詞組補部。
指詞組可以爲縛詞組補部。
縛詞組可以爲謂詞組補部。
謂詞組可以爲態詞組補部、可以爲謂詞組冠部。
態詞組可以爲助詞組補部。
助詞組可以爲成詞組補部、可以爲助詞組冠部。
成詞組可以爲體詞組補部、可以爲成詞組冠部。
人之所思也成詞組可以以懷。
核部必治補部與其所有、補部必治核部與其所有。
冠部必治心部與其所有、心部必冠核部與其所有。
詞有引補部之核者、有引其所治詞組者、有刪補部之核者、有刪其所治詞組者。


\chapter*{辭書}
\setlength\columnsep{24pt}
\begin{multicols}{2}
\onehalfspacing
\flushleft
\input{leed-entries.tex}
\end{multicols}
\chapter*{文法}
\onehalfspacing
文法也者,文人所依組字者也。

\begin{forest} 
[動字組
    [定部\\指字組
        [核部\\指字 \zero]
        [補部\\名字組 露酒]]
    [修部\\副字組 [一]]
    [核部\\動字 [進]]
    [補部\\介字組 [於酒家]]]
\end{forest}
\chapter*{謝恩}
\onehalfspacing
余所能以編此書者以有應結草所報之恩也。余始編此書之時,猶未逾大學本科生。故,余所未曾學者頗多,看書而頻借其所語,不語其所由焉。

余生於槿邦,幼時雖習漢字於初校,猶未懷志才,而中學有恩師姜氏者執日用漢語教太白詩而動吾心,有恩師朴氏者執文言而矯余病文。

因故退於高校,妄思此身之分,圖進首爾大學,而習高校漢文於教育放送金夫子。若未蒙其教,詩文類童必為左右笑。
雖竭力而習漢文而遠東文、英文、算學之類斯謀重舉,而竟不為余所行。
時,橫海覽扶桑欲讀和文於京都大學,而國事之亂奪此志矣。

余所以修中國語學者徒以天之鞏此志也。
以是讀於韓國放送通信大學。緣日本學部門之不教古典文學,乃修中國語言文學。
於是謝之如此。
\end{document}

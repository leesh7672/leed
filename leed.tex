\documentclass[a5paper,10pt]{report}
\usepackage{luatexko}
\usepackage{geometry}
\usepackage{multicol}
\usepackage{setspace}
\usepackage{wrapfig}
\usepackage{xcolor}
\usepackage{graphicx}
\usepackage{etoolbox}
\usepackage{titlesec}
\usepackage{zhnumber}
\usepackage{cancel}
\usepackage[linguistics]{forest}
\usepackage{tikz}

\titleformat{\chapter}{\fontsize{18pt}{18pt}\selectfont}{}{0pt}{}
\titleformat{\section}{\fontsize{14pt}{14pt}\selectfont}{}{0pt}{}

\definecolor{gold}{rgb}{0.4,0.3,0.2}
\definecolor{deepgold}{rgb}{0.3,0.2,0.1}
\graphicspath{{./images/}}

\hanjabyhanjafont=1
\setmainfont[InterCharStretch=2pt,Vertical=Alternates,RawFeature=vertical]{HCR Batang Lvt}
\setmainhangulfont[InterCharStretch=2pt,Vertical=Alternates,RawFeature=vertical]{HCR Batang Lvt}
\setmainhanjafont[InterCharStretch=2pt,Vertical=Alternates,RawFeature=vertical]{HCR Batang Lvt}

\newcommand{\zero}{∅}
\newcommand{\explain}[2]{{$\bullet$}〔{#1}〕{#2}}
\newcommand{\entry}[4]{\par\textcolor{gold}{#1}\textsuperscript{#2} {#3}}
\newcommand{\cref}[1]{\textcolor{gold}{#1}}

\geometry{top=10mm,left=10mm,bottom=25mm,right=10mm}
\verticaltypesetting
\begin{document}
\begin{titlepage}
\hfill
\vfill
\hspace{-40mm}
\includegraphics[angle=90,width=250mm]{cover.png}
\vfill
{\fontsize{14pt}{14pt}\selectfont\textcolor{gold}{李世鎬}\hspace{24pt}編}
\vspace{24pt}\newline
{\fontsize{36pt}{36pt}\selectfont\textcolor{gold}{\hspace{48pt}古典漢文敎學辭書}}
\vfill
\end{titlepage}
\newpage
\chapter*{首}
\onehalfspacing
凡華夏・雞林・扶桑・中山・安南之類、
昔者士夫齊嘗識漢文、用諸書塾・官廳・江湖矣。
故今日該地多童應於校習漢文焉。
漢文重雖如此、槿原之童都爲之無所以當識者頗久、窮究者甚少。
今者諸國徒以其京辭爲通言之本、
舉筆則書口舌之所言如茶飯、而不識高尙古文。
嗚呼、今天下人人輕習英語、以字典日新也。
若欲習漢文也、今猶看古者所寶字典也。
然今所有字典雖說善詞彙之義、而不足以識其所以用也。
吾不勝爲之惜然、而敢編此書。
此心雖丹、而獨作字書之業頗苦焉、斯求諸賢之志於天下。
請欲參之者見 http://github.com/leesh7672/leed。
此書之瑕無參之者爲誰、徒以吾不察也爾。
各詞或取諸漢文古籍、有從古道所作者慮日用裏、勸自作也。
詞彙乃順康熙字典、乃慮天下同胞。
同源之詞同項而登、異源之詞別項而載。
此所以與漢文釋詞義而析其用者、欲使習之者精通文理也。
夢思此書之爲江湖諸賢所修及天下諸儒所寶、
心不遠千里而行。


\chapter*{辭書}
\setlength\columnsep{24pt}
\begin{multicols}{2}
\onehalfspacing
\flushleft
\input{leed-entries.tex}
\end{multicols}
\chapter*{文法}
\onehalfspacing
文法也者,文人所依組字者也。

\begin{forest} 
[動字組
    [定部\\指字組
        [核部\\指字 \zero]
        [補部\\名字組 露酒]]
    [修部\\副字組 [一]]
    [核部\\動字 [進]]
    [補部\\介字組 [於酒家]]]
\end{forest}
\chapter*{謝恩}
\onehalfspacing
此所以編此書者、以有應結草所報之恩也。
始編此書之時、猶未過本科生。
故愚所未修學藝甚蕃、看書而頻借其所語、不語其所由焉。
生於槿邦、幼時雖習漢字於初校、未懷志才也。
然中學有恩師姜氏者執日用漢語、敎太白詩而動吾心、有恩師朴氏者執漢文、而矯吾文。
因故退於高校、不知此身之分、圖進首都大學、而習高校漢文於敎養放送金夫子。
若未蒙其敎、詩文類童、必爲左右笑。
雖盡心習漢文、而遠東文・英文・算學之類、謀重修、而不擧。
日横海遊覽扶桑、欲讀和文於京都大學、而國事之亂奪其志矣。
以是讀於韓國放送通信大學、因日本學科不教古文學、乃修中語中文學。
小生修中語學、徒以天道竊所行也。
是以謝之如此。

\end{document}

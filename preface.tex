\chapter*{序文}
夫單詞也者、空詞與充詞之大名。空詞也者、單詞之未備字與音而具義。充詞也者、單詞之均有字與音與義。
漢文單詞品以體詞・謂詞・狀詞・束詞・補詞・助詞・歎詞。
體詞咸語生物・死物・地物・時物。古分體詞以名詞・數詞・量詞、其能不分、名之以體詞也。
體詞可成主部・賓部・補部・狀部。
謂詞皆語體詞所語之態。古分謂詞以動詞與形容詞、其能不分、名以謂詞也。
謂詞維謂部寔成。
狀詞都爲定部・狀部。一統單詞・單節・單句・複句者曰束詞。語單節之格者曰補詞。助詞也者、語氣之謂也。
助詞結單句與單句所有之部。
夫單句應有謂部焉。
單句先有謂部、後可從謂部而具主部・狀部・賓部・補部。
單句先備主部・賓部・補部、後有得其定部。
若單句先有狀部、後有其狀部。
單句一有補部、宜裝其賓部也。
主部與狀部置於謂部之前、賓部與補部放於謂部之前。
定部槪在其前、而其長也、可在其後。
如已所道、單詞有空詞者、無其文字、故內文不之登載。然不之識、難以善讀、而記如此。
空詞有空體詞・空謂詞・空束詞・空補詞・空助詞之類。
除空助詞外、空詞必不在單句之尾。
空體詞之所言、輕可以知。空謂詞、有輕可以知者、時語主部之必當補部。
束詞皆可以空束詞代之、補詞亦齊可以空補詞代之、助詞咸可以空助詞代之。
謂部之所語之係於存在・顯現也、可以空補詞爲補部、以語謂部所說之物者取其賓部也。

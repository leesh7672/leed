\chapter*{序文}
華夏・雞林・扶桑・南越之中、書生齊嘗與漢文流通。
夫欲習漢文也、古以字典爲寶及今。
字典雖有字彙之釋、而未有詞。
各國有其辭書、以諺釋文、
不識其國之諺、畫中之餅也爾。
內文各項或取諸漢籍、或從古道而新製。
各項順之康熙字典。
同源之詞同項而登、異源之詞別項而載。
以文釋文、詳述其用、弱通漢文、皆可以讀。
此書之爲江湖諸賢所修與天下諸儒所寶甚遙遠。
\chapter*{文法}
單詞品以名詞・指詞・定詞・動詞・介詞・連詞
・量詞・數詞・首詞・助詞・狀詞。
單句析以首部・主部・狀部・述部・賓部・補部・定部。

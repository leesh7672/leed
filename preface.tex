\chapter*{序文}
華夏・雞林・扶桑・南越之中、書生齊曾通漢文矣。
夫欲習漢文也、古以字典爲寶及今。
字典雖有字彙之釋、而未有詞。
各國有其辭書、以諺釋文、
不識其國之諺、畫中之餅也爾。
內文各項或取諸漢籍、或從古道而新製。
各項順之康熙字典。
同源之詞同項而登、異源之詞別項而載。
以文釋文、詳述其用、弱通漢文、皆可以讀。
此書之爲江湖諸賢所修與天下諸儒所寶甚遙遠。
\chapter*{文法}
詞品以體詞・算詞・數詞・指詞・結詞・介詞・縛詞・謂詞・態詞・氣詞・成詞。
\cref{父}・\cref{犬}・\cref{葡萄}・\cref{石}之類、謂之體詞。
\cref{又}・\cref{有}之類、謂之算詞。
\cref{二}・\cref{旬}・\cref{萬}之類、謂之數詞。
\cref{此}・\cref{堯}・\cref{誰}之類、謂之指詞。
\cref{與}・\cref{及}・\cref{兼}之類、謂之結詞。
\cref{於}・\cref{猶}・\cref{曰}之類、謂之介詞。
\cref{動}・\cref{爲}・\cref{作}・\cref{去}・\cref{有}・\cref{美}・\cref{因}之類、謂之謂詞。
\cref{以}・\cref{使}・\cref{遭}之類、謂之態詞。
\cref{也}・\cref{矣}之類、謂之氣詞。
\cref{凡}・\cref{若}・\cref{雖}・\cref{而}之類、謂之成詞。

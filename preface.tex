\chapter*{序文}
華夏・雞林・扶桑・南越之中、書生齊嘗與漢文流通。
夫欲習漢文也、古以字典爲寶及今。
字典雖有字彙之釋、而未有詞。
各國有其辭書、以諺釋文、
不識其國之諺、畫中之餅也爾。
內文各項或取諸漢籍、或從古道而新製。
各項順之康熙字典。
同源之詞同項而登、異源之詞別項而載。
以文釋文、詳述其用、弱通漢文、皆可以讀。
此書之爲江湖諸賢所修與天下諸儒所寶甚遙遠。
\chapter*{文法}
詞品以成詞・助詞・狀詞・態詞・謂詞・縛詞・結詞・指詞
・質詞・體詞・度詞・算詞・數詞之大名也。
\cref{父}・\cref{堯}・\cref{犬}・\cref{葡萄}・\cref{石}之類曰體詞。
\cref{二}・\cref{旬}・\cref{百}・\cref{萬}之類曰數詞。
\cref{又}・\cref{有}之類、曰算詞。
\cref{日}・\cref{月}・\cref{匹}之類曰度詞。
\cref{此}・\cref{何}・\cref{誰}之類曰指詞。
\cref{與}・\cref{及}・\cref{兼}之類曰結詞。
\cref{於}・\cref{猶}・\cref{曰}之類曰縛詞。
\cref{爲}・\cref{去}・\cref{有}・\cref{美}・\cref{因}之類曰謂詞。
\cref{相}・\cref{使}・\cref{遭}之類曰態詞。
\cref{也}・\cref{矣}・\cref{乎}・\cref{而}之類曰助詞。
\cref{大凡}・\cref{若}・\cref{雖}・\cref{然}・\cref{而}之類曰成詞。
\cref{不}・\cref{皆}・\cref{徒}・\cref{所}・\cref{以}之類曰狀詞。
\cref{上}・\cref{下}之類曰狀詞。

大凡詞組時析以核部與補部、時復析以冠部與心部、心部之成似詞組。
詞組與心部之內先有冠部、然後心部。
非助詞組與度詞組也、先有核部、後有補部。
助詞組與度詞組也、先有補部、後有核部。
數詞可以爲數詞組核部、算詞可以爲算詞組核部、度詞可以爲度詞組核部
束詞可以爲束詞組核部、體詞可以爲體詞組核部、指詞可以爲指詞組核部、
結詞可以爲結詞組核部、縛詞可以爲縛詞組核部、謂詞可以爲謂詞組核部、
態詞可以爲態詞組核部、助詞可以爲助詞組核部、成詞可以爲成詞組核部。
質詞可以爲體詞組冠部、狀詞可以爲態詞組冠部。

數詞組可以爲算詞組補部、可以爲度詞組補部。
算詞組可以爲數詞組補部。
度詞組可以爲體詞組補部。
體詞組可以爲指詞組補部、可以爲結詞組冠部。
結詞組可以爲指詞組補部。
指詞組可以爲縛詞組補部。
縛詞組可以爲謂詞組補部、可以爲縛詞組補部。
謂詞組可以爲態詞組補部、可以爲謂詞組冠部。
態詞組可以爲助詞組補部。
助詞組可以爲成詞組補部、可以爲助詞組冠部。
成詞組可以爲體詞組補部、可以爲成詞組冠部。
人之所思也成詞組可以以懷。
核部必治補部與其所有、補部必治核部與其所有。
冠部必治心部與其所有、心部必冠核部與其所有。
詞有引補部之核者、有引其所治詞組者、有刪補部之核者、有刪其所治詞組者。

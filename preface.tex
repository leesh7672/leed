凡禹域九州、槿原八道、日本八洲、球璃三山之內,夙昔士夫咸識漢文,用之於書院、官廳、江湖。
故少童迄今應習之於校焉。
今者諸國徒以京辭為通言,文章都似口舌所言如茶飯,而不識高尚古文。
然人人輕習英語,以字典日新也。
如欲習漢文也,迄今檢字於古者所寶字典也。
然今日所有之字典雖說字之所指,而不足以教其所以用也。
余不勝為之惜然,而敢編此書也。
此書如有瑕,其所以然者,徒以余不察也爾。
各字或取諸文言古籍,有從古道而新作者。
余所以新作之者,欲利乎日用之中,以勸自作也。
字字順康熙字典,乃慮天下同胞。
此所以與漢文釋詞義而析其用者,欲使習之者精通文理也。
夢中思此書之為江湖諸賢所修,而為天下諸儒所寶,
心斯不遠萬里而行焉。

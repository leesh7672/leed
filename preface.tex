\chapter*{序文}
華夏・雞林・扶桑・南越之中、書生齊嘗與漢文流通。
夫欲習漢文也、古以字典爲寶及今。
字典雖有字彙之釋、而未有詞。
各國有其辭書、以諺釋文、
不識其國之諺、畫中之餅也爾。
內文各項或取諸漢籍、或從古道而新製。
各項順之康熙字典。
同源之詞同項而登、異源之詞別項而載。
以文釋文、詳述其用、弱通漢文、皆可以讀。
此書之爲江湖諸賢所修與天下諸儒所寶甚遙遠。
\chapter*{文法}
詞品以首詞・助詞・狀詞・態詞・謂詞・縛詞・結詞・指詞
・質詞・體詞・度詞・算詞・數詞之大名也。
\cref{父}・\cref{堯}・\cref{犬}・\cref{葡萄}・\cref{石}之類曰體詞。
\cref{二}・\cref{旬}・\cref{百}・\cref{萬}之類曰數詞。
\cref{又}・\cref{有}之類、曰算詞。
\cref{日}・\cref{月}・\cref{匹}之類曰度詞。
\cref{此}・\cref{何}・\cref{誰}之類曰指詞。
\cref{與}・\cref{及}・\cref{兼}之類曰結詞。
\cref{於}・\cref{猶}・\cref{曰}之類曰縛詞。
\cref{動}・\cref{爲}・\cref{去}・\cref{有}・\cref{美}・\cref{因}之類曰謂詞。
\cref{相}・\cref{使}・\cref{遭}之類曰態詞。
\cref{也}・\cref{矣}・\cref{乎}之類曰助詞。
\cref{凡}・\cref{若}・\cref{雖}・\cref{而}之類曰首詞。
\cref{不}・\cref{未}・\cref{皆}・\cref{徒}・\cref{所}・\cref{以}之類曰狀詞。
大凡短語時析以核部與補部、時復析以冠部與心部、心部之成似短語。
短語與心部之內先有冠部、然後心部。
助詞短語與度詞短語之外、先有核部、後有補部。
助詞短語與度詞短語之內、先有補部、後有核部。
數詞足爲數詞短語核部、算詞足爲算詞短語核部、度詞足爲度詞短語核部
束詞足爲束詞短語核部、體詞足爲體詞短語核部、指詞足爲指詞短語核部、
結詞足爲結詞短語核部、縛詞足爲縛詞短語核部、謂詞足爲謂詞短語核部、
態詞足爲態詞短語核部、助詞足爲助詞短語核部、首詞足爲首詞短語核部。
質詞足爲體詞短語冠部、狀詞足爲態詞短語冠部。
數詞短語足爲算詞短語補部、足爲度詞短語補部。
算詞短語足爲數詞短語補部。
度詞短語足爲體詞短語補部。
體詞短語足爲指詞短語補部。
指詞短語足爲結詞短語補部、足爲縛詞短語補部。
縛詞短語足爲謂詞短語補部。
謂詞短語足爲態詞短語補部。
態詞短語足爲助詞短語補部。
首詞短語足爲體詞短語補部。
人之所思也首詞短語足以懷。
核部必治補部與其所有、補部必治核部與其所有。
冠部必治心部與其所有、心部必冠核部與其所有。
詞時有引核者引核部於補部、時有引品者思品引短語於其所治。

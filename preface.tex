\section{首}
華夏、雞林、扶桑之類,古以漢語爲國風之礎矣。
古欲習漢文,以字典爲寶。
字典雖有字彙之解、未有詞。
各國,雖有其詞書,以諺釋文。
不識其國之諺,畫中之餅也耳。
以文釋文,進釋四海之諺,詳述其用,弱通漢文,皆可以讀。
余願此爲江湖諸賢所補,爲四海諸學所寶耳。
\section{範}
此書之範如左。
\par 第一條:「各詞,或取諸漢語,或古道而新製。」
\par 第一條:「各項,順之康熙字典。」
\par 第三條:「無雅無俗,以文解之。」
\par 第四條:「變品之詞,如從常道,不之載。」
\par 第五條:「同源同字同音之詞,同項而載。否則分項而載。」

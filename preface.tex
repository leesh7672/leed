凡禹域九州・槿原八道・日本八洲・球璃三山之內,
夙昔士夫咸識漢文,用諸書院、官廳、江湖。
故今日該地多童應習漢文於校焉。
漢文雖重如此,槿原之童都爲之無所以當識者頗久,窮究者甚少。
今者諸國徒以該國京辭與英語爲通言、
文章都似口舌所言如茶飯,而不識高尙古文。
嗚呼,今天下人人輕習英語,以字典日新也。
今欲習漢文也,迄檢字於古者所寶辭典也。
然今日所有辭典雖說辭彙之義,而不足以教其所以用也。
余不勝爲之惜然,而敢編此書。
此書所以有瑕者,徒以余不察也爾。
各詞或取諸漢文古籍,有從古道而新作者欲利日用之中,以勸自作也。
詞彙乃順康熙字典,乃慮天下同胞。
此所以與漢文釋詞義而析其用者,欲使習之者精通文理也。
夢思此書之爲江湖諸賢所修及天下諸儒所寶,
心不遠千里而行。

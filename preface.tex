禹域、槿原、扶桑、萬春之內,士夫夙昔咸識華夏文言而用之於書院、官廳、江湖。
故,少童迄今應學之於校。

然至近者,諸國徒通國音,只寫口語。
于是,識高尙古文者歲少。雖然,羣衆易英語,其以字典日新也。

如欲習唐詩・宋文,迄今檢字於古者所寶之字典焉。
今所有夏文字典雖說字之所指而不足以敎其所以用也。

余惜之而敢編此書也。
字字有取諸文言古籍者、有從古道而新作者。
余所以新作之者以欲利之於日用之中以勸自作也。
字字順康熙字典,乃慮天下同胞。

此所以徒用文言釋字義、狀其用者以欲使習之者精通文理也。
此書如有瑕,徒以余不察也。

夢思此書之為江湖諸賢所修,而爲天下諸儒所寶,心斯不遠萬里而行焉。

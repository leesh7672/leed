\chapter*{序文}
凡華夏・雞林・扶桑・中山・安南之類、
昔者高尙士齊嘗識漢文、通用於書塾・官廳・江湖矣。
故該地多童應於校習漢文焉。
漢文重雖如此、槿原之童都爲之無所以當識之物頗久。
今者諸國徒以其京辭爲通言之本、
舉筆則書口舌之所言如茶飯。
若欲習漢文也、古之所寶字典及今也。
然今所有字典雖說善詞彙之意、而不足以識其所以用也。
吾不勝爲之惜然、而敢作此書。
此心雖丹、而獨作字書之業頗苦焉、斯求諸賢之志於天下。
或欲參之、請見 https://github.com/leesh7672/leed。
此書之瑕無參之者爲誰、徒以吾不察也爾。
內文各詞或取諸漢籍、或從古道而新作、而順康熙字典。
同源之詞同項而登、異源之詞別項而載。
釋以漢文、詳述其用、弱通漢文、皆可以用。
夢思此書之爲江湖諸賢所修及天下諸儒所寶、
心情乃千里行。

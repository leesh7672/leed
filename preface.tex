禹域・槿原・扶桑・萬春之內、士夫夙昔咸識華夏文言而用之於學館・官衙・精舍・山亭。
故幼童至今日還當學諸中等校。

然諸國徒通國音只寫口語。
于是識高尙古文者歲少。
雖然、易學英語者之日而多。
長幼之所以易習英文者、其以字典時而新也。
如欲習唐詩與宋文、猶檢字於古者所寶之字典焉。
今所有夏文字典雖說字之所指而不足以敎其所以用也。

余惜之而敢編此書。
字字有取諸文言古籍者、亦有從古道而新作者。
余所以新作之者以欲利之於日用之中以勸自作也。
字字順康熙字典乃慮天下同胞。
此所以徒用文言釋字義而狀其用者以欲使習之者精通文理也。
此書如有瑕、徒以余不察也。

夢思此書之爲江湖諸賢所修而爲天下諸儒所寶、心斯不遠萬里而行焉。

凡九州禹域・八道雞林・八洲日本・三山球璃之內、
夙昔士夫齊嘗識漢文、用諸書院・官廳・江湖矣。
故今日該地多童應於校習漢文焉。
漢文重雖如此、槿原之童都爲之無所以當識者頗久、窮究者甚少。
今者諸國徒以其京辭爲通言之本、
舉筆則書口舌之所言如茶飯、而不識高尙古文。
嗚呼、今天下人人輕習英語、以字典日新也。
若欲習漢文也、今猶看古者所寶字典也。
然今所有字典雖說善詞彙之義、而不足以識其所以用也。
余不勝爲之惜然、而敢編此書。
此書所以有瑕者、徒以余不察也爾。
各詞或取諸漢文古籍、有從古道而新作者、欲利日用之中、以勸自作也。
詞彙乃順康熙字典、乃慮天下同胞。
同形同音之詞同項而登、異源之詞別項而載。
此所以與漢文釋詞義而析其用者、欲使習之者精通文理也。
夢思此書之爲江湖諸賢所修及天下諸儒所寶、
心不遠千里而行。

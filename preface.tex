\chapter*{序文}
華夏・雞林・扶桑・南越之中、書生齊嘗與漢文流通。
夫欲習漢文也、古以字典爲寶及今。
字典雖有字彙之釋、而未有詞。
各國有其辭書、以諺釋文、
不識其國之諺、畫中之餅也爾。
內文各項或取諸漢籍、或從古道而新製。
各項順之康熙字典。
同源之詞同項而登、異源之詞別項而載。
以文釋文、詳述其用、弱通漢文、皆可以讀。
此書之爲江湖諸賢所修與天下諸儒所寶甚遙遠。
\chapter*{文法}
詞品以首詞・助詞・態詞・謂詞・縛詞・結詞・指詞・質詞・體詞・度詞・算詞・數詞之大名也。
\cref{父}・\cref{堯}・\cref{犬}・\cref{葡萄}・\cref{石}之類、曰體詞。
\cref{二}・\cref{旬}・\cref{百}・\cref{萬}之類、曰數詞。
\cref{又}・\cref{之}之類、曰算詞。
\cref{日}・\cref{月}・\cref{匹}之類、曰度詞。
\cref{此}・\cref{何}・\cref{誰}之類、曰指詞。
\cref{與}・\cref{及}・\cref{兼}・\cref{之}之類、曰結詞。
\cref{於}・\cref{猶}・\cref{曰}之類、曰縛詞。
\cref{動}・\cref{爲}・\cref{去}・\cref{有}・\cref{美}・\cref{因}之類、曰謂詞。
\cref{相}・\cref{使}・\cref{遭}・\cref{所}之類、曰態詞。
\cref{也}・\cref{矣}・\cref{乎}之類、曰助詞。
\cref{凡}・\cref{若}・\cref{雖}・\cref{而}之類、曰首詞。

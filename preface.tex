凡禹域、槿原、扶桑、萬春之內夙昔士夫咸識華夏文言,用於書院、官廳、江湖,故少童迄今應習諸校焉。
今者諸國徒以國音為通言,文章都似口舌所言如茶飯,識高尚古文者歲少,然人人輕習英語,以字典日新也。
如欲習唐詩、宋文也,迄今檢字於古者所寶之字典,然今所有夏文字典雖說字之所指,而不足以教其所以用也。
余不勝為之惜然,而敢編此書也。各字或取諸文言古籍,有從古道而新作者。余所以新作之者,欲利之於日用之中以勸自作也。
字字順康熙字典,乃慮天下同胞。此所以用文言釋字義而析其用者,欲使習之者精通文理也。
此書如有瑕,其所以然者,徒以余不察也。夢思此書之為江湖諸賢所修,而為天下諸儒所寶,心斯不遠萬里而行焉。

\chapter*{序文}
華夏・雞林・扶桑・南越之中、書生齊曾通漢文矣。
夫欲習漢文也、古以字典爲寶及今。
字典雖有字彙之釋、而未有詞。
各國有其辭書、以諺釋文,
不識其國之諺、畫中之餅也爾。
內文各項或取諸漢籍、或從古道而新製。
各項順之康熙字典。
同源之詞同項而登、異源之詞別項而載。
詞分以體詞・算詞・數詞・指詞・結詞・介詞・謂詞・態詞・氣詞・成詞・歎詞。
以文釋文、詳述其用、弱通漢文、皆可以讀。
此書之爲江湖諸賢所修與天下諸儒所寶甚遙遠。

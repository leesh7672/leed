\documentclass[a5paper,landscape,10pt]{memoir}
\usepackage{fontspec}
\usepackage[top=10mm,left=10mm,bottom=10mm,right=10mm]{geometry}
\usepackage{multicol}
\usepackage{setspace}
\usepackage[nospace]{xeCJK}
\usepackage{CJKnumb}
\usepackage{nameref}
\usepackage{makeidx}
\pagestyle{empty}
\newcommand*\CJKmovesymbol[1]{\raise.35em\hbox{#1}}
\newcommand*\CJKmove{\punctstyle{plain}
		       \let\CJKsymbol\CJKmovesymbol
			   \let\CJKpunctsymbol\CJKsymbol}
\setmainfont[RawFeature=vertical,Vertical=RotatedGlyphs]{Noto Serif CJK KR}
\setCJKmainfont[RawFeature=vertical,Vertical=RotatedGlyphs]{Noto Serif CJK KR}
\newcommand{\explain}[2]{\\{{#1}}\\{$\bullet$}{#2}}
\newcommand{\go}[2]{\\{$\bullet$}請考\textbf{#2} 矣。 }
\newcommand{\entry}[4]{\par{\begin{minipage}{0.2\textwidth}{{\vspace{2mm}\hspace{4mm}{\bfseries{#1}}{\textsuperscript{\CJKnumber{#2}}}{#3}}}\end{minipage}}\\}
\newcommand{\also}[1]{\\或書之\textbf{#1}。}
\newcommand{\samp}[2]{{#1}云、「{#2}」}
\newcommand{\syn}[2]{與\textbf{#1}\textsuperscript{\CJKnumber{#2}}相通。}
\newcommand{\ant}[2]{與\textbf{#1}\textsuperscript{\CJKnumber{#2}}相對。}
\begin{document}
\frontmatter
\hspace{0pt}
\vfill
{\vspace{14pt}\fontsize{48pt}{0pt}{漢詞林}}
\vfill
\hspace{18pt}
\newpage
\linespread{1.5}
凡華夏・雞林・扶桑・中山・安南之類、
昔者士夫齊嘗識漢文、用諸書塾・官廳・江湖矣。
故今日該地多童應於校習漢文焉。
漢文重雖如此、槿原之童都爲之無所以當識者頗久、窮究者甚少。
今者諸國徒以其京辭爲通言之本、
舉筆則書口舌之所言如茶飯、而不識高尙古文。
嗚呼、今天下人人輕習英語、以字典日新也。
若欲習漢文也、今猶看古者所寶字典也。
然今所有字典雖說善詞彙之義、而不足以識其所以用也。
吾不勝爲之惜然、而敢編此書。
此心雖丹、而獨作字書之業頗苦焉、斯求諸賢之志於天下。
請欲參之者見 http://github.com/leesh7672/leed。
此書之瑕無參之者爲誰、徒以吾不察也爾。
各詞或取諸漢文古籍、有從古道所作者慮日用裏、勸自作也。
詞彙乃順康熙字典、乃慮天下同胞。
同源之詞同項而登、異源之詞別項而載。
此所以與漢文釋詞義而析其用者、欲使習之者精通文理也。
夢思此書之爲江湖諸賢所修及天下諸儒所寶、
心不遠千里而行。

\mainmatter
\linespread{1.5}
\begin{multicols}{4}
\input{hcl-entries.tex}
\end{multicols}
\end{document}

\documentclass[a5paper,12pt]{article}
\usepackage{luatexja-fontspec}
\usepackage[top=-10mm,left=10mm,bottom=30mm,right=10mm]{geometry}
\usepackage{multicol}
\usepackage{setspace}
\usepackage{wrapfig}
\usepackage{xcolor}
\usepackage{graphicx}
\usepackage[style=chicago-authordate]{biblatex}

\pagestyle{empty}
\addbibresource{hcl}

\definecolor{blue}{rgb}{0.2,0.15,0.1}
\definecolor{deepblue}{rgb}{0.4,0.3,0.2}
\graphicspath{{./images/}}

\setmainfont{TimesNewerRoman}
\setsansfont{TimesNewerRoman}
\setromanfont{TimesNewerRoman}
\setmainfont[Script=CJK]{AsebiMinchoLight}
\setsansfont[Script=CJK]{AsebiMinchoLight}
\setromanfont[Script=CJK]{AsebiMinchoLight}

\newcommand{\CJKnumber}[1]{{#1}}
\newcommand{\explain}[2]{{$\bullet$}〔{#1}〕{#2}}
\newcommand{\go}[2]{\\{$\bullet$}請考\textcolor{blue}{#2}。 }
\newcommand{\entry}[4]{{\par{{\vspace{1.5mm}{\textcolor{deepblue}{#1}}{\textsuperscript{\CJKnumber{#2}}}{#3}}}}}
\newcommand{\also}[1]{\\亦曰\textcolor{blue}{#1}。}
\newcommand{\samp}[2]{{#1}曰:『{#2}』}
\newcommand{\syn}[2]{通\textcolor{blue}{#1}\textsuperscript{\CJKnumber{#2}}。}
\newcommand{\ant}[2]{對\textcolor{blue}{#1}\textsuperscript{\CJKnumber{#2}}。}

\begin{document}
\linespread{1.25}
\begin{wrapfigure}{r}{60mm}
\includegraphics[height=220mm]{cover.png}
\end{wrapfigure}
\hfill
\vfill
{\fontsize{32}{32}\textcolor{deepblue}{漢詞林}}\\
{\textcolor{blue}{李世鎬}\hspace{14pt}編著}
\vspace{64pt}
\newpage
\addtolength{\topmargin}{20mm}
\linespread{1.25}
凡華夏・雞林・扶桑・中山・安南之類、
昔者士夫齊嘗識漢文、用諸書塾・官廳・江湖矣。
故今日該地多童應於校習漢文焉。
漢文重雖如此、槿原之童都爲之無所以當識者頗久、窮究者甚少。
今者諸國徒以其京辭爲通言之本、
舉筆則書口舌之所言如茶飯、而不識高尙古文。
嗚呼、今天下人人輕習英語、以字典日新也。
若欲習漢文也、今猶看古者所寶字典也。
然今所有字典雖說善詞彙之義、而不足以識其所以用也。
吾不勝爲之惜然、而敢編此書。
此心雖丹、而獨作字書之業頗苦焉、斯求諸賢之志於天下。
請欲參之者見 http://github.com/leesh7672/leed。
此書之瑕無參之者爲誰、徒以吾不察也爾。
各詞或取諸漢文古籍、有從古道所作者慮日用裏、勸自作也。
詞彙乃順康熙字典、乃慮天下同胞。
同源之詞同項而登、異源之詞別項而載。
此所以與漢文釋詞義而析其用者、欲使習之者精通文理也。
夢思此書之爲江湖諸賢所修及天下諸儒所寶、
心不遠千里而行。


\section{詞典}
\input{hcl-entries.tex}

\appendix
\printbibliography

\end{document}
